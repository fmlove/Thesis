% Options for packages loaded elsewhere
\PassOptionsToPackage{unicode}{hyperref}
\PassOptionsToPackage{hyphens}{url}
%
\documentclass[
  12pt,
  a4paper,
]{article}
\usepackage{amsmath,amssymb}
\usepackage{lmodern}
\usepackage{setspace}
\usepackage{iftex}
\ifPDFTeX
  \usepackage[T1]{fontenc}
  \usepackage[utf8]{inputenc}
  \usepackage{textcomp} % provide euro and other symbols
\else % if luatex or xetex
  \usepackage{unicode-math}
  \defaultfontfeatures{Scale=MatchLowercase}
  \defaultfontfeatures[\rmfamily]{Ligatures=TeX,Scale=1}
\fi
% Use upquote if available, for straight quotes in verbatim environments
\IfFileExists{upquote.sty}{\usepackage{upquote}}{}
\IfFileExists{microtype.sty}{% use microtype if available
  \usepackage[]{microtype}
  \UseMicrotypeSet[protrusion]{basicmath} % disable protrusion for tt fonts
}{}
\makeatletter
\@ifundefined{KOMAClassName}{% if non-KOMA class
  \IfFileExists{parskip.sty}{%
    \usepackage{parskip}
  }{% else
    \setlength{\parindent}{0pt}
    \setlength{\parskip}{6pt plus 2pt minus 1pt}}
}{% if KOMA class
  \KOMAoptions{parskip=half}}
\makeatother
\usepackage{xcolor}
\usepackage[margin=1in]{geometry}
\usepackage{color}
\usepackage{fancyvrb}
\newcommand{\VerbBar}{|}
\newcommand{\VERB}{\Verb[commandchars=\\\{\}]}
\DefineVerbatimEnvironment{Highlighting}{Verbatim}{commandchars=\\\{\}}
% Add ',fontsize=\small' for more characters per line
\usepackage{framed}
\definecolor{shadecolor}{RGB}{248,248,248}
\newenvironment{Shaded}{\begin{snugshade}}{\end{snugshade}}
\newcommand{\AlertTok}[1]{\textcolor[rgb]{0.94,0.16,0.16}{#1}}
\newcommand{\AnnotationTok}[1]{\textcolor[rgb]{0.56,0.35,0.01}{\textbf{\textit{#1}}}}
\newcommand{\AttributeTok}[1]{\textcolor[rgb]{0.77,0.63,0.00}{#1}}
\newcommand{\BaseNTok}[1]{\textcolor[rgb]{0.00,0.00,0.81}{#1}}
\newcommand{\BuiltInTok}[1]{#1}
\newcommand{\CharTok}[1]{\textcolor[rgb]{0.31,0.60,0.02}{#1}}
\newcommand{\CommentTok}[1]{\textcolor[rgb]{0.56,0.35,0.01}{\textit{#1}}}
\newcommand{\CommentVarTok}[1]{\textcolor[rgb]{0.56,0.35,0.01}{\textbf{\textit{#1}}}}
\newcommand{\ConstantTok}[1]{\textcolor[rgb]{0.00,0.00,0.00}{#1}}
\newcommand{\ControlFlowTok}[1]{\textcolor[rgb]{0.13,0.29,0.53}{\textbf{#1}}}
\newcommand{\DataTypeTok}[1]{\textcolor[rgb]{0.13,0.29,0.53}{#1}}
\newcommand{\DecValTok}[1]{\textcolor[rgb]{0.00,0.00,0.81}{#1}}
\newcommand{\DocumentationTok}[1]{\textcolor[rgb]{0.56,0.35,0.01}{\textbf{\textit{#1}}}}
\newcommand{\ErrorTok}[1]{\textcolor[rgb]{0.64,0.00,0.00}{\textbf{#1}}}
\newcommand{\ExtensionTok}[1]{#1}
\newcommand{\FloatTok}[1]{\textcolor[rgb]{0.00,0.00,0.81}{#1}}
\newcommand{\FunctionTok}[1]{\textcolor[rgb]{0.00,0.00,0.00}{#1}}
\newcommand{\ImportTok}[1]{#1}
\newcommand{\InformationTok}[1]{\textcolor[rgb]{0.56,0.35,0.01}{\textbf{\textit{#1}}}}
\newcommand{\KeywordTok}[1]{\textcolor[rgb]{0.13,0.29,0.53}{\textbf{#1}}}
\newcommand{\NormalTok}[1]{#1}
\newcommand{\OperatorTok}[1]{\textcolor[rgb]{0.81,0.36,0.00}{\textbf{#1}}}
\newcommand{\OtherTok}[1]{\textcolor[rgb]{0.56,0.35,0.01}{#1}}
\newcommand{\PreprocessorTok}[1]{\textcolor[rgb]{0.56,0.35,0.01}{\textit{#1}}}
\newcommand{\RegionMarkerTok}[1]{#1}
\newcommand{\SpecialCharTok}[1]{\textcolor[rgb]{0.00,0.00,0.00}{#1}}
\newcommand{\SpecialStringTok}[1]{\textcolor[rgb]{0.31,0.60,0.02}{#1}}
\newcommand{\StringTok}[1]{\textcolor[rgb]{0.31,0.60,0.02}{#1}}
\newcommand{\VariableTok}[1]{\textcolor[rgb]{0.00,0.00,0.00}{#1}}
\newcommand{\VerbatimStringTok}[1]{\textcolor[rgb]{0.31,0.60,0.02}{#1}}
\newcommand{\WarningTok}[1]{\textcolor[rgb]{0.56,0.35,0.01}{\textbf{\textit{#1}}}}
\usepackage{graphicx}
\makeatletter
\def\maxwidth{\ifdim\Gin@nat@width>\linewidth\linewidth\else\Gin@nat@width\fi}
\def\maxheight{\ifdim\Gin@nat@height>\textheight\textheight\else\Gin@nat@height\fi}
\makeatother
% Scale images if necessary, so that they will not overflow the page
% margins by default, and it is still possible to overwrite the defaults
% using explicit options in \includegraphics[width, height, ...]{}
\setkeys{Gin}{width=\maxwidth,height=\maxheight,keepaspectratio}
% Set default figure placement to htbp
\makeatletter
\def\fps@figure{htbp}
\makeatother
\setlength{\emergencystretch}{3em} % prevent overfull lines
\providecommand{\tightlist}{%
  \setlength{\itemsep}{0pt}\setlength{\parskip}{0pt}}
\setcounter{secnumdepth}{-\maxdimen} % remove section numbering
\newlength{\cslhangindent}
\setlength{\cslhangindent}{1.5em}
\newlength{\csllabelwidth}
\setlength{\csllabelwidth}{3em}
\newlength{\cslentryspacingunit} % times entry-spacing
\setlength{\cslentryspacingunit}{\parskip}
\newenvironment{CSLReferences}[2] % #1 hanging-ident, #2 entry spacing
 {% don't indent paragraphs
  \setlength{\parindent}{0pt}
  % turn on hanging indent if param 1 is 1
  \ifodd #1
  \let\oldpar\par
  \def\par{\hangindent=\cslhangindent\oldpar}
  \fi
  % set entry spacing
  \setlength{\parskip}{#2\cslentryspacingunit}
 }%
 {}
\usepackage{calc}
\newcommand{\CSLBlock}[1]{#1\hfill\break}
\newcommand{\CSLLeftMargin}[1]{\parbox[t]{\csllabelwidth}{#1}}
\newcommand{\CSLRightInline}[1]{\parbox[t]{\linewidth - \csllabelwidth}{#1}\break}
\newcommand{\CSLIndent}[1]{\hspace{\cslhangindent}#1}
\usepackage{fancyhdr}
\ifLuaTeX
  \usepackage{selnolig}  % disable illegal ligatures
\fi
\IfFileExists{bookmark.sty}{\usepackage{bookmark}}{\usepackage{hyperref}}
\IfFileExists{xurl.sty}{\usepackage{xurl}}{} % add URL line breaks if available
\urlstyle{same} % disable monospaced font for URLs
\hypersetup{
  hidelinks,
  pdfcreator={LaTeX via pandoc}}

\author{}
\date{\vspace{-2.5em}}

\begin{document}

\setstretch{1.15}
\hypertarget{er-ergic-and-golgi-in-axons}{%
\section{ER, ERGIC, and Golgi in
axons}\label{er-ergic-and-golgi-in-axons}}

\chaptermark{ER, ERGIC, and Golgi}

\hypertarget{introduction}{%
\subsection{Introduction}\label{introduction}}

The discovery of local secretory pathways in neurons is fairly recent,
but has already contributed significantly to our understanding of how
these cells develop and function. In non-polarised cells, protein
translation and modification typically take place near the nucleus, and
proteins are then transported to their final destination in, on, or
outside of the cell. While this does also occur in neurons, neurons are
also capable of locally translating proteins, which can then be
delivered to organelles or inserted into the cell membrane as needed
{[}1{]}. It has become increasingly apparent that protein delivery from
the cell body is not sufficient to maintain protein turnover at distal
parts of the cell, nor is it fast enough to support the dynamic
behaviour observed in these regions, for example during axon
pathfinding. Instead, these regions rely on several overlapping local
pathways for protein production, modification, and secretion (reviewed
in {[}2{]}).

\hypertarget{er-exit-sites-and-cop-vesicles}{%
\subsubsection{ER exit sites and COP
vesicles}\label{er-exit-sites-and-cop-vesicles}}

Key to all of these pathways is the presence of interconnected ER
tubules throughout the neuron. Continuous with the ER network in the
soma, these tubules reach into dendrites, into individual spines in an
activity-dependent manner {[}3{]}, and down the entirety of the axon.
Among other roles, this ER network provides a platform for local protein
translation, and targeting to other organelles for further modification.

Proteins translated in the ER are exported via ER exit sites (ERES),
typically in structures known as COPII vesicles. Recent work as
demonstrated that these exit sites are complex tubular structures,
formed by the interactions of numerous COPII-associated proteins
{[}4{]}. A number of different chaperones and cargo receptors are
involved in this process, sorting proteins for export and facilitating
the budding of these transport intermediates from the ER network.
{[}5{]} showed that Sar1, a GTPase related to COPII budding, was
targeted to mammalian axons during neuronal development, and that this
was essential for axonal growth.

\hypertarget{the-er-golgi-intermediate-compartment}{%
\subsubsection{The ER-Golgi intermediate
compartment}\label{the-er-golgi-intermediate-compartment}}

After budding from the ER, COPII vesicles fuse a dynamic structure known
as the ER-Golgi intermediate compartment (ERGIC) (reviewed in {[}6{]}).
From here, cargo can be directed to different downstream organelles for
further processing.

\hypertarget{golgi-outposts-and-satellites}{%
\subsubsection{Golgi outposts and
satellites}\label{golgi-outposts-and-satellites}}

The most common destination after leaving the ERGIC is the Golgi,
although this can take different forms depending on the region of the
cell involved. In the cell body, this is typically the perinuclear Golgi
stack. In dendrites, proteins may be targeted to Golgi outposts or Golgi
satellites. While these are structurally distinct, they both carry out
essential Golgi functions, including protein glycosylation {[}7{]}.
Recently, Golgi satellites have also been identified in axons, both in
static and dynamically transported populations. Here, they can act as
local glycosylation platforms, similar to in dendrites, but also take
part in delivering functional proteins to lysosomes {[}8{]} and nerve
terminals {[}9{]}. Golgi satellite localisation, and by extension their
functionality, is dynamically regulated by neuronal activity {[}10{]}.

\hypertarget{further-processing-and-protein-secretion}{%
\subsubsection{Further processing and protein
secretion}\label{further-processing-and-protein-secretion}}

After leaving a Golgi-related organelle, proteins are often sorted in
endosomes through the retromer complex, which returns cargo receptors to
the Golgi. This complex has been identified in dendrites closely
associated with Golgi satellites, suggesting that this sorting also
occurs as part of local secretory pathways {[}7{]}. Recent work has
linked retromer dysfunction to a number of neurodegenerative conditions,
potentially through changes to the lysosomal proteome {[}11{]}.

After sorting by the retromer complex, proteins can then be delivered to
other organelles or the plasma membrane. Several studies have found
local translation, processing, and secretion of cell surface proteins,
including voltage-gated sodium channels at nodes of Ranvier {[}12{]}.
The exocyst complex facilitates fusion of post-Golgi carriers with the
plasma membrane, and interference with this complex has been shown to
inhibit neurite growth {[}13,14{]}.

\hypertarget{the-golgi-bypass-pathway}{%
\subsubsection{The Golgi-bypass
pathway}\label{the-golgi-bypass-pathway}}

An alternate, Golgi-bypassing pathway has also been suggested, with
cargo moving from the ERGIC to the cell surface via recycling endosomes.
This has been studied in dendrites, where it was found to play a role in
delivering GluA1 glutamate receptors to dendritic spines {[}15{]}. This
study does not fully rule out the involvement of Golgi satellites, but
does show that this pathway is robust to the disruption of somatic Golgi
and Golgi outposts with brefeldin A.

\hypertarget{methods}{%
\subsection{Methods}\label{methods}}

\hypertarget{fluorescence-quantification}{%
\subsubsection{Fluorescence
quantification}\label{fluorescence-quantification}}

Immunochemistry was carried out as described in @ref\{METHODS\}. Tracing
of regions of interest was assisted by a custom ImageJ macro, which is
available at (GITHUB REF). The area of the nucleus was excluded from
cell body quantification. Background correction was carried out for axon
terminals, axons, and dendrites by subtracting the intensity of an
identical ROI, manually shifted to avoid overlap the area of interest.
Cell bodies were not background-corrected, due to brighter overall
intensity and difficulty in identifying a nearby `background' region
free of neurites or other cellular structures.

\hypertarget{inhibition-of-er-to-golgi-transport}{%
\subsubsection{Inhibition of ER to Golgi
transport}\label{inhibition-of-er-to-golgi-transport}}

We tested several inhibitors of either ER to Golgi transport or Golgi
function in DIV3 cortical neurons --- H-89 (100\(\mu\)M)
\emph{(Calbiochem, 371962)}, clofibrate (500\(\mu\)M)
\emph{(Sigma-Aldrich, C6643)}, brefeldin A (10\(\mu\)g/mL)
\emph{(Calbiochem, 500583)}, and golgicide A \emph{(Calbiochem, 345862)}
(10\(\mu\)M). These concentrations were based on previously published
use {[}4,12,16--18{]}. Concentration-matched DMSO-only controls were
used for each inhibitor. The inhibitors were added to cells for 30
minutes at 37°, 7\% CO\textsubscript{2}, in supplement-free media, after
which cells were fixed. Given protrudin's effect on ERGIC localisation
reported below @ref\{fig:ERGIC\_plot\}, we used ERGIC53 staining as
measure of inhibitor function. We found that H-89 caused growth cone
collapse, while the others preserved cellular morphology, and clofibrate
led to a sight decrease in ERGIC53 signal in both the cell body and
growth cone (data not shown).

Based on these preliminary results, we decided to test of clofibrate
could block protrudin's pro-regenerative effect on injured axons.
Unfortunately, this dosage of clofibrate proved highly toxic over the 16
hour timescale required for this experiment. We were unable to optimise
this experiment further, as our regular supply of rats for cortical
neuron cultures was disrupted by an animal carriage embargo at Frankfurt
airport during the summer of 2022. We plan to return to this in a future
project.

\hypertarget{results}{%
\subsection{Results}\label{results}}

\hypertarget{er-golgi-pathway-localisation}{%
\subsubsection{ER-Golgi pathway
localisation}\label{er-golgi-pathway-localisation}}

We first set about to determine the baseline localisation of our
selected markers, and how this changed during normal neuronal
development. We stained cortical neurons for BCAP31, SURF4, ERGIC53, and
SLC38A1 and DIV3 and DIV15, and compared fluorescence intensity between
the cell body and axon terminal. This showed that, with the exception of
SURF4, all markers show a relative decline at the axon terminal with age
@ref\{fig:dev-changed-full-figure\}. Separate analysis of each region
shows that all markers show increased intensity in the cell body with
age, but that most show no significant change in the axon terminal, with
the exception of SLC38A1, which increases. (SUPPLEMENTAL FIGURE?) This
is in keeping with SLC38A1's role as an amino acid transporter, which is
a particularly important function for synaptically active neurons
{[}19{]}. It is worth noting that, while the average intensity remains
the same, the axon terminal dramatically reduces in size as the cell
matures, so this represents a decrease in the total amount of these
proteins present at the axon tip.

\begin{Shaded}
\begin{Highlighting}[]
\NormalTok{dev\_ratio\_plot}
\end{Highlighting}
\end{Shaded}

\begin{figure}
\centering
\includegraphics{../figures/full/dev-changed-full-figure-1.png}
\caption{A) Ratio of fluorescence intensity between Cell body and axon
terminal in DIV3 vs DIV15 neurons. (TODO - N PER CONDITION) B-M)
Representative images of DIV3 neurons (B-E), DIV15 cell bodies (F-I),
and DIV15 axon terminals (J-M) All scale bars are 20\(\mu\)m, and J-M
are 20x20\(\mu\)m. Markers correspond to the graph at the top of each
column.}
\end{figure}

The staining was repeated in DIV1 (post-differentiation) i3 neurons, to
confirm that the axon terminal:cell body ratio was conserved between
rodent and human neurons. This showed remarkable consistency between
cell types, with the exception of BCAP31, which had a very slightly
higher ratio in rat neurons.

\begin{Shaded}
\begin{Highlighting}[]
\NormalTok{i3\_plot}
\end{Highlighting}
\end{Shaded}

\begin{figure}
\centering
\includegraphics{../figures/full/i3-ratio-plot-1.png}
\caption{A) Ratio of fluorescence intensity between Cell body and axon
terminal in DIV1 i3 vs DIV3 rat cortical neurons. (TODO - N PER
CONDITION) B-E) Representative images of DIV1 i3 neurons stained for
each marker. All scale bars are 20\(\mu\)m.}
\end{figure}

We also tested a tool called ESCargo(FTV) to visualise ER exit sites
{[}10{]}. This encodes a SURF4-dependent synthetic secretory cargo,
which can be inducibly released from the ER. While the aggregates were
clearly visible in cell bodies, we only observed them in a small subset
of axons, and even then only very sparsely along their length
@ref\{fig:ESCargo\_fig\}. In some cases, these were found at axonal
swellings or branch points, raising the possibility that these exit
sites may concentrate at particular structural features. Further work is
required to test if this corresponds to the localisation of functional
ER exit sites in the axon, although it seems unlikely to represent the
full picture, given how previous studies have demonstrated the
importance of local protein translation for diverse axonal functions.

\begin{Shaded}
\begin{Highlighting}[]
\NormalTok{ESCargo\_plot}
\end{Highlighting}
\end{Shaded}

\begin{figure}
\centering
\includegraphics{../figures/full/ESCargo-figure-1.png}
\caption{A) Cell body of a cell co-expressing EGFP/Emerald with
ESCargo(FTV)-Crimson. Note individual aggregates or small clusters in
dendrites, often at branch points. B) Axon of neuron depicted in A, with
two ESCargo(FTV) aggregates circled. All scale bars are 20\(\mu\)m.}
\end{figure}

\hypertarget{the-role-of-protrudin-in-localisation-and-transport}{%
\subsubsection{The role of protrudin in localisation and
transport}\label{the-role-of-protrudin-in-localisation-and-transport}}

Given the developmental decline in these markers at axon terminals, we
were interested to see if protrudin might act by increasing the axonal
levels of ER-Golgi pathway components. We repeated the experiment above
in DIV15 neurons expressing either EGFP or EGFP-tagged protrudin, and
measured the fluorescence intensity in the cell body and axon terminals.

\hypertarget{bcap31}{%
\paragraph{BCAP31}\label{bcap31}}

BCAP31 showed high variability between cells, particularly in the soma.
This may have been a result of signal amplifcation with a biotinylated
secondary antibody, although the smaller range of intensities measured
at axon terminals suggests that this may represent natural variation. In
any case, while protrudin did appear to decrease cell body BCAP31
levels, this did not translate to a change in the axon terminal:cell
body ratio. Therefore, we do not believe that protrudin substantially
changes the distribution of BCAP31 in cells. (Figure
@ref(fig:BCAP31-plot))

\begin{Shaded}
\begin{Highlighting}[]
\NormalTok{BCAP31\_plot}
\end{Highlighting}
\end{Shaded}

\begin{figure}
\centering
\includegraphics{../figures/fullBCAP31-plot-1.png}
\caption{A-F) Representative images of BCAP31 in control, WT, and active
protrudin-expressing cell bodies (A-C, respectively) and axon terminals
(D-F). All scale bars are 20\(\mu\)m, and D-F are 20x20\(\mu\)m. G-I)
Quantification of BCAP31 fluorescence intensity. (TODO - N PER
CONDITION)}
\end{figure}

Peculiarly, in many cases BCAP31 signal appeared higher in cells
expressing either protrudin or control EGFP, compared to untransfected
neighbouring cells. It is possible that the levels of this protein are
dynamicaly controlled in response to the cell's overall level of protein
synthesis, which may be impacted by exogenous plasmids. This should be
taken into account in any future experiments examining BCAP31 levels or
localisation.

\hypertarget{surf4}{%
\paragraph{SURF4}\label{surf4}}

SURF4 decreased both in the cell body and axon terminal of cells
expressing active protrudin. However, these decreases were well-matched,
resulting in no change to the axon terminal:cell body ratio. It is
possible that protrudin is not changing the localisation of this ERES
protein, but may alter its overall expression levels. (Figure
@ref(fig:SURF4-plot))

\begin{Shaded}
\begin{Highlighting}[]
\NormalTok{SURF4\_plot}
\end{Highlighting}
\end{Shaded}

\begin{figure}
\centering
\includegraphics{../figures/fullSURF4-plot-1.png}
\caption{A-F) Representative images of SURF4 in control, WT, and active
protrudin-expressing cell bodies (A-C, respectively) and axon terminals
(D-F). All scale bars are 20\(\mu\)m, and D-F are 20x20\(\mu\)m. G-I)
Quantification of SURF4 fluorescence intensity. (TODO - N PER
CONDITION)}
\end{figure}

\hypertarget{slc38a1}{%
\paragraph{SLC38A1}\label{slc38a1}}

SLC38A1 showed no significant changes with either WT or active protrudin
expression, although there was a slight trend towards higher axonal
intensity. (Figure @ref(fig:SLC38A1-plot))

\begin{Shaded}
\begin{Highlighting}[]
\NormalTok{SLC38A1\_plot}
\end{Highlighting}
\end{Shaded}

\begin{figure}
\centering
\includegraphics{../figures/fullSLC38A1-plot-1.png}
\caption{A-F) Representative images of SLC38A1 in control, WT, and
active protrudin-expressing cell bodies (A-C, respectively) and axon
terminals (D-F). All scale bars are 20\(\mu\)m, and D-F are
20x20\(\mu\)m. G-I) Quantification of SLC38A1 fluorescence intensity.
(TODO - N PER CONDITION)}
\end{figure}

\hypertarget{ergic53}{%
\paragraph{ERGIC53}\label{ergic53}}

In contrast with the other markers, ERGIC53 showed significantly altered
localisation with the expression of active, but not wild type,
protrudin. This included a significant increase in axon terminal
intensity and axon terminal:cell body ratio, on top of a slight increase
in soma intensity. This suggests that protrudin is playing a role in
either the formation of ERGIC compartments, or their transport into the
distal axon. (Figure @ref(fig:ERGIC53-plot))

\begin{Shaded}
\begin{Highlighting}[]
\NormalTok{ERGIC53\_plot}
\end{Highlighting}
\end{Shaded}

\begin{figure}
\centering
\includegraphics{../figures/fullERGIC53-plot-1.png}
\caption{A-F) Representative images of ERGIC53 in control, WT, and
active protrudin-expressing cell bodies (A-C, respectively) and axon
terminals (D-F). All scale bars are 20\(\mu\)m, and D-F are
20x20\(\mu\)m. G-I) Quantification of ERGIC53 fluorescence intensity.
(TODO - N PER CONDITION)}
\end{figure}

\hypertarget{protrudins-effect-on-golgi-satellites}{%
\subsubsection{Protrudin's effect on Golgi
satellites}\label{protrudins-effect-on-golgi-satellites}}

Given protrudin's effect on the ERGIC compartment, we were particularly
interested in seeing if this also impacted axonal Golgi satellites. As
these are highly dynamic organelles, we decided to look at Golgi
satellite transport in the presence of protrudin. We co-expressed
mCherry-tagged protrudin constructs with St3Gal5-EGFP, as a marker of
Golgi satellites.

\begin{Shaded}
\begin{Highlighting}[]
\NormalTok{St3Gal5\_kymo\_plot}
\end{Highlighting}
\end{Shaded}

\begin{figure}
\centering
\includegraphics{../figures/full/St3Gal-full-figure-1.png}
\caption{A) Representative axon image and kymographs showing St3Gal5
transport in the distal axon of cortical neurons expressing either
mCherry (control) or mCherry-tagged protrudin constructs. B) Total
number of St3Gal5-labelled organelles in each cell imaged. C) Average
run length (per cell) of moving Golgi satellites Anterograde and
retrograde were defined as continuous movements of at least 5µm away
from or towards the cell body, respectively. D) Average Golgi satellite
velocity (per cell).}
\end{figure}

Following live imaging, these cells were fixed, and imaged at higher
exposure on a confocal microscope to more clearly visualise Golgi
satellites in the axon. We measured both the number of puncta, as well
as point intensity of each, as a proxy for size. This confirmed that
protrudin overexpression did not affect the overall number of Golgi
satellites in the distal axon. This

\begin{Shaded}
\begin{Highlighting}[]
\NormalTok{GS\_puncta\_full\_figure}
\end{Highlighting}
\end{Shaded}

\begin{figure}
\centering
\includegraphics{../figures/full/GS-puncta-full-figure-1.png}
\caption{A) Representative axon image showing St3Gal5 puncta. Scale bar
is 20\(\mu\)m. B) Number of puncta per length of axon imaged. C)Point
fluorescence intensity of each St3Gal3 puncta counted. n = 18 cells per
condition across three independent batches.}
\end{figure}

We then conducted a scratch assay to examine the localisation of Golgi
satellites in the growth cones of mature neurons. Cortical neurons
cultured on glass coverslips were scratched with a sterile needle, then
incubated for 18-20 hours before fixing. Axons growing into the
scratched region were imaged on a confocal microscope.

This revealed a strong concentration of St3Gal5 at the base of each
growth cone, with slightly weaker signal extending further distally.
Where St3Gal5 was co-expressed with protrudin, it is apparent that the
Golgi satellite signal reaches further into the growth cone than the ER,
suggesting secretory machinery oriented towards the direction of growth.

\begin{Shaded}
\begin{Highlighting}[]
\NormalTok{scratch\_fig}
\end{Highlighting}
\end{Shaded}

\begin{figure}
\centering
\includegraphics{../figures/full/scratch-assay-full-1.png}
\caption{Representative images of growth cones on DIV15 cortical neurons
co-expressing either mCherry (A) or mCherry-protrudin (B-C) with
St3Gal5-EGFP. All scale bars are 20\(\mu\)m.}
\end{figure}

\hypertarget{discussion}{%
\subsection{Discussion}\label{discussion}}

These results provide an insight into how protrudin might regulate local
secretory pathways in the axon. While it does not appear to change the
distribution of ERES-associated proteins, such as BCAP31 and SURF4,
active protrudin does increase the relative level of ERGIC53 in axon
terminals. As the ERGIC is a distinct compartment from the ER, this
suggests a secretory role for protrudin beyond its effects on the axonal
ER. It is not yet clear if this represents an increased formation of
ERGIC from ERES-derived vesicles in the axon, or increased transport of
these structures from elsewhere in the cell. However, it does clearly
show that protrudin overexpression, and in particular active protrudin,
can impact the local secretory machinery present in the axon, which is
likely to play a role in protrudin's neuroprotective and regenerative
effects.

This is further supported by our finding that active protrudin affects
Golgi satellite transport in the distal axon. Unexpectedly, we find
slower transport of Golgi satellites in this case, particularly in the
anterograde direction, despite unchanged run length or overall number of
organelles. Further experiments are required to identify a mechanism for
this change, although it could represent a sustained interaction with
another organelle during transport, potentially ERGIC or a related
compartment.

These results may account for the finding that active protrudin has a
stronger regenerative effect than WT protrudin, but that the two
constructs do not significantly differ in Rab11 or integrin transport
{[}20{]}.

\hypertarget{refs}{}
\begin{CSLReferences}{0}{0}
\leavevmode\vadjust pre{\hypertarget{ref-meriandaAfunctionalequivalent2009}{}}%
\CSLLeftMargin{1. }%
\CSLRightInline{Merianda TT, Lin AC, Lam JSY, Vuppalanchi D, Willis DE,
Karin N, et al. A functional equivalent of endoplasmic reticulum and
{Golgi} in axons for secretion of locally synthesized proteins.
Molecular and Cellular Neuroscience. 2009;40: 128--142.
doi:\href{https://doi.org/10.1016/j.mcn.2008.09.008}{10.1016/j.mcn.2008.09.008}}

\leavevmode\vadjust pre{\hypertarget{ref-gonzalezGolgibypassreview2018}{}}%
\CSLLeftMargin{2. }%
\CSLRightInline{González C, Cornejo VH, Couve A. Golgi bypass for local
delivery of axonal proteins, fact or fiction? Current Opinion in Cell
Biology. 2018;53: 9--14.
doi:\href{https://doi.org/10.1016/j.ceb.2018.03.010}{10.1016/j.ceb.2018.03.010}}

\leavevmode\vadjust pre{\hypertarget{ref-perez-alvarezEndoplasmicReticulumVisits2020}{}}%
\CSLLeftMargin{3. }%
\CSLRightInline{Perez-Alvarez A, Yin S, Schulze C, Hammer JA, Wagner W,
Oertner TG. Endoplasmic reticulum visits highly active spines and
prevents runaway potentiation of synapses. Nature Communications.
2020;11: 5083.
doi:\href{https://doi.org/10.1038/s41467-020-18889-5}{10.1038/s41467-020-18889-5}}

\leavevmode\vadjust pre{\hypertarget{ref-weigelERtoGolgiProteinDelivery2021}{}}%
\CSLLeftMargin{4. }%
\CSLRightInline{Weigel AV, Chang C-L, Shtengel G, Xu CS, Hoffman DP,
Freeman M, et al. {ER-to-Golgi} protein delivery through an interwoven,
tubular network extending from {ER}. Cell. 2021;184: 2412--2429.e16.
doi:\href{https://doi.org/10.1016/j.cell.2021.03.035}{10.1016/j.cell.2021.03.035}}

\leavevmode\vadjust pre{\hypertarget{ref-aridorSelectiveTargetingER2009}{}}%
\CSLLeftMargin{5. }%
\CSLRightInline{Aridor M, Fish KN. Selective {Targeting} of {ER Exit
Sites Supports Axon Development}. Traffic. 2009;10: 1669--1684.
doi:\href{https://doi.org/10.1111/j.1600-0854.2009.00974.x}{10.1111/j.1600-0854.2009.00974.x}}

\leavevmode\vadjust pre{\hypertarget{ref-appenzeller-herzogERGolgiIntermediateCompartment2006}{}}%
\CSLLeftMargin{6. }%
\CSLRightInline{Appenzeller-Herzog C, Hauri H-P. The {ER-Golgi}
intermediate compartment ({ERGIC}): In search of its identity and
function. Journal of Cell Science. 2006;119: 2173--2183.
doi:\href{https://doi.org/10.1242/jcs.03019}{10.1242/jcs.03019}}

\leavevmode\vadjust pre{\hypertarget{ref-mikhaylovaDendriticGolgiSatellite2016}{}}%
\CSLLeftMargin{7. }%
\CSLRightInline{Mikhaylova M, Bera S, Kobler O, Frischknecht R, Kreutz
MR. A {Dendritic Golgi Satellite} between {ERGIC} and {Retromer}. Cell
Reports. 2016;14: 189--199.
doi:\href{https://doi.org/10.1016/j.celrep.2015.12.024}{10.1016/j.celrep.2015.12.024}}

\leavevmode\vadjust pre{\hypertarget{ref-liePostGolgicarriers2021}{}}%
\CSLLeftMargin{8. }%
\CSLRightInline{Lie PPY, Yang D-S, Stavrides P, Goulbourne CN, Zheng P,
Mohan PS, et al. Post-{Golgi} carriers, not lysosomes, confer lysosomal
properties to pre-degradative organelles in normal and dystrophic axons.
Cell Reports. 2021;35: 109034.
doi:\href{https://doi.org/10.1016/j.celrep.2021.109034}{10.1016/j.celrep.2021.109034}}

\leavevmode\vadjust pre{\hypertarget{ref-cornejoNonconventionalAxonalOrganelles2020}{}}%
\CSLLeftMargin{9. }%
\CSLRightInline{Cornejo VH, González C, Campos M, Vargas-Saturno L,
Juricic M de los Á, Miserey-Lenkei S, et al. Non-conventional {Axonal
Organelles Control TRPM8 Ion Channel Trafficking} and {Peripheral Cold
Sensing}. Cell Reports. 2020;30: 4505--4517.e5.
doi:\href{https://doi.org/10.1016/j.celrep.2020.03.017}{10.1016/j.celrep.2020.03.017}}

\leavevmode\vadjust pre{\hypertarget{ref-govindActivitydependentGolgiSatellite2021}{}}%
\CSLLeftMargin{10. }%
\CSLRightInline{Govind AP, Jeyifous O, Russell TA, Yi Z, Weigel AV,
Ramaprasad A, et al. Activity-dependent {Golgi} satellite formation in
dendrites reshapes the neuronal surface glycoproteome. Bard FA, Malhotra
V, Yu F, Hanus C, editors. eLife. 2021;10: e68910.
doi:\href{https://doi.org/10.7554/eLife.68910}{10.7554/eLife.68910}}

\leavevmode\vadjust pre{\hypertarget{ref-dalyMultiomicApproachCharacterises2022}{}}%
\CSLLeftMargin{11. }%
\CSLRightInline{Daly JL, Danson CM, Lewis PA, Riccardo S, Filippo LD,
Cacchiarelli D, et al. Multiomic {Approach Characterises} the
{Neuroprotective Role} of {Retromer} in {Regulating Lysosomal Health}.
{bioRxiv}; 2022. p. 2022.09.13.507260.
doi:\href{https://doi.org/10.1101/2022.09.13.507260}{10.1101/2022.09.13.507260}}

\leavevmode\vadjust pre{\hypertarget{ref-gonzalezAxonsprovidesecretory2016}{}}%
\CSLLeftMargin{12. }%
\CSLRightInline{González C, Cánovas J, Fresno J, Couve E, Court FA,
Couve A. Axons provide the secretory machinery for trafficking of
voltage-gated sodium channels in peripheral nerve. Proceedings of the
National Academy of Sciences. 2016;113: 1823--1828.
doi:\href{https://doi.org/10.1073/pnas.1514943113}{10.1073/pnas.1514943113}}

\leavevmode\vadjust pre{\hypertarget{ref-pereiraExocystComplexEssential2022}{}}%
\CSLLeftMargin{13. }%
\CSLRightInline{Pereira C, Stalder D, Anderson G, Shun-Shion AS,
Houghton J, Antrobus R, et al. The exocyst complex is an essential
component of the mammalian constitutive secretory pathway. {bioRxiv};
2022. p. 2022.05.26.493223.
doi:\href{https://doi.org/10.1101/2022.05.26.493223}{10.1101/2022.05.26.493223}}

\leavevmode\vadjust pre{\hypertarget{ref-swopeExocystComplexRequired2022}{}}%
\CSLLeftMargin{14. }%
\CSLRightInline{Swope RD, Hertzler JI, Stone MC, Kothe GO, Rolls MM. The
exocyst complex is required for developmental and regenerative neurite
growth in vivo. Developmental Biology. 2022;492: 1--13.
doi:\href{https://doi.org/10.1016/j.ydbio.2022.09.005}{10.1016/j.ydbio.2022.09.005}}

\leavevmode\vadjust pre{\hypertarget{ref-bowenGolgiindependentSecretoryTrafficking2017}{}}%
\CSLLeftMargin{15. }%
\CSLRightInline{Bowen AB, Bourke AM, Hiester BG, Hanus C, Kennedy MJ.
Golgi-independent secretory trafficking through recycling endosomes in
neuronal dendrites and spines. Davis GW, editor. eLife. 2017;6: e27362.
doi:\href{https://doi.org/10.7554/eLife.27362}{10.7554/eLife.27362}}

\leavevmode\vadjust pre{\hypertarget{ref-helmsInhibitionBrefeldinGolgi1992}{}}%
\CSLLeftMargin{16. }%
\CSLRightInline{Helms JB, Rothman JE. Inhibition by brefeldin {A} of a
{Golgi} membrane enzyme that catalyses exchange of guanine nucleotide
bound to {ARF}. Nature. 1992;360: 352--354.
doi:\href{https://doi.org/10.1038/360352a0}{10.1038/360352a0}}

\leavevmode\vadjust pre{\hypertarget{ref-defigueiredoClofibrateInhibitsMembrane1999a}{}}%
\CSLLeftMargin{17. }%
\CSLRightInline{de Figueiredo P, Brown WJ. Clofibrate inhibits membrane
trafficking to the {Golgi} complex and induces its retrograde movement
to the endoplasmic reticulum. Cell Biology and Toxicology. 1999;15:
311--323.
doi:\href{https://doi.org/10.1023/A:1007667802497}{10.1023/A:1007667802497}}

\leavevmode\vadjust pre{\hypertarget{ref-saenzGolgicideRevealsEssential2009}{}}%
\CSLLeftMargin{18. }%
\CSLRightInline{Sáenz JB, Sun WJ, Chang JW, Li J, Bursulaya B, Gray NS,
et al. Golgicide {A} reveals essential roles for {GBF1} in {Golgi}
assembly and function. Nature Chemical Biology. 2009;5: 157--165.
doi:\href{https://doi.org/10.1038/nchembio.144}{10.1038/nchembio.144}}

\leavevmode\vadjust pre{\hypertarget{ref-hellstenNeuronalAstrocyticProtein2017}{}}%
\CSLLeftMargin{19. }%
\CSLRightInline{Hellsten SV, Hägglund MG, Eriksson MM, Fredriksson R.
The neuronal and astrocytic protein {SLC38A10} transports glutamine,
glutamate, and aspartate, suggesting a role in neurotransmission. FEBS
Open Bio. 2017;7: 730--746.
doi:\href{https://doi.org/10.1002/2211-5463.12219}{10.1002/2211-5463.12219}}

\leavevmode\vadjust pre{\hypertarget{ref-petrovaProtrudinFunctionsEndoplasmic2020}{}}%
\CSLLeftMargin{20. }%
\CSLRightInline{Petrova V, Pearson CS, Ching J, Tribble JR, Solano AG,
Yang Y, et al. Protrudin functions from the endoplasmic reticulum to
support axon regeneration in the adult {CNS}. Nature Communications.
2020;11: 5614.
doi:\href{https://doi.org/10.1038/s41467-020-19436-y}{10.1038/s41467-020-19436-y}}

\end{CSLReferences}

\end{document}
