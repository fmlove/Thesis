% Options for packages loaded elsewhere
\PassOptionsToPackage{unicode}{hyperref}
\PassOptionsToPackage{hyphens}{url}
%
\documentclass[
]{article}
\usepackage{amsmath,amssymb}
\usepackage{lmodern}
\usepackage{iftex}
\ifPDFTeX
  \usepackage[T1]{fontenc}
  \usepackage[utf8]{inputenc}
  \usepackage{textcomp} % provide euro and other symbols
\else % if luatex or xetex
  \usepackage{unicode-math}
  \defaultfontfeatures{Scale=MatchLowercase}
  \defaultfontfeatures[\rmfamily]{Ligatures=TeX,Scale=1}
\fi
% Use upquote if available, for straight quotes in verbatim environments
\IfFileExists{upquote.sty}{\usepackage{upquote}}{}
\IfFileExists{microtype.sty}{% use microtype if available
  \usepackage[]{microtype}
  \UseMicrotypeSet[protrusion]{basicmath} % disable protrusion for tt fonts
}{}
\makeatletter
\@ifundefined{KOMAClassName}{% if non-KOMA class
  \IfFileExists{parskip.sty}{%
    \usepackage{parskip}
  }{% else
    \setlength{\parindent}{0pt}
    \setlength{\parskip}{6pt plus 2pt minus 1pt}}
}{% if KOMA class
  \KOMAoptions{parskip=half}}
\makeatother
\usepackage{xcolor}
\usepackage[margin=1in]{geometry}
\usepackage{longtable,booktabs,array}
\usepackage{calc} % for calculating minipage widths
% Correct order of tables after \paragraph or \subparagraph
\usepackage{etoolbox}
\makeatletter
\patchcmd\longtable{\par}{\if@noskipsec\mbox{}\fi\par}{}{}
\makeatother
% Allow footnotes in longtable head/foot
\IfFileExists{footnotehyper.sty}{\usepackage{footnotehyper}}{\usepackage{footnote}}
\makesavenoteenv{longtable}
\usepackage{graphicx}
\makeatletter
\def\maxwidth{\ifdim\Gin@nat@width>\linewidth\linewidth\else\Gin@nat@width\fi}
\def\maxheight{\ifdim\Gin@nat@height>\textheight\textheight\else\Gin@nat@height\fi}
\makeatother
% Scale images if necessary, so that they will not overflow the page
% margins by default, and it is still possible to overwrite the defaults
% using explicit options in \includegraphics[width, height, ...]{}
\setkeys{Gin}{width=\maxwidth,height=\maxheight,keepaspectratio}
% Set default figure placement to htbp
\makeatletter
\def\fps@figure{htbp}
\makeatother
\setlength{\emergencystretch}{3em} % prevent overfull lines
\providecommand{\tightlist}{%
  \setlength{\itemsep}{0pt}\setlength{\parskip}{0pt}}
\setcounter{secnumdepth}{-\maxdimen} % remove section numbering
\newlength{\cslhangindent}
\setlength{\cslhangindent}{1.5em}
\newlength{\csllabelwidth}
\setlength{\csllabelwidth}{3em}
\newlength{\cslentryspacingunit} % times entry-spacing
\setlength{\cslentryspacingunit}{\parskip}
\newenvironment{CSLReferences}[2] % #1 hanging-ident, #2 entry spacing
 {% don't indent paragraphs
  \setlength{\parindent}{0pt}
  % turn on hanging indent if param 1 is 1
  \ifodd #1
  \let\oldpar\par
  \def\par{\hangindent=\cslhangindent\oldpar}
  \fi
  % set entry spacing
  \setlength{\parskip}{#2\cslentryspacingunit}
 }%
 {}
\usepackage{calc}
\newcommand{\CSLBlock}[1]{#1\hfill\break}
\newcommand{\CSLLeftMargin}[1]{\parbox[t]{\csllabelwidth}{#1}}
\newcommand{\CSLRightInline}[1]{\parbox[t]{\linewidth - \csllabelwidth}{#1}\break}
\newcommand{\CSLIndent}[1]{\hspace{\cslhangindent}#1}
\usepackage{booktabs}
\usepackage{longtable}
\usepackage{array}
\usepackage{multirow}
\usepackage{wrapfig}
\usepackage{float}
\usepackage{colortbl}
\usepackage{pdflscape}
\usepackage{tabu}
\usepackage{threeparttable}
\usepackage{threeparttablex}
\usepackage[normalem]{ulem}
\usepackage{makecell}
\usepackage{xcolor}
\ifLuaTeX
  \usepackage{selnolig}  % disable illegal ligatures
\fi
\IfFileExists{bookmark.sty}{\usepackage{bookmark}}{\usepackage{hyperref}}
\IfFileExists{xurl.sty}{\usepackage{xurl}}{} % add URL line breaks if available
\urlstyle{same} % disable monospaced font for URLs
\hypersetup{
  hidelinks,
  pdfcreator={LaTeX via pandoc}}

\author{}
\date{\vspace{-2.5em}}

\begin{document}

\hypertarget{identifying-protrudins-neuronal-functions}{%
\section{Identifying protrudin's neuronal
functions}\label{identifying-protrudins-neuronal-functions}}

%\chaptermark{Neuronal functions}

\hypertarget{introduction}{%
\subsection{Introduction}\label{introduction}}

As discussed previously in section @ref(intro-ptdn), protrudin's array
of binding domains allow it to participate in numerous cellular systems,
with diverse effects. To identify other mechanisms of action, we decided
to take a broad approach, examining a number of different systems to
identify how protrudin might influence neuronal function, either
endogenously or with overexpression.

Previous work in the lab found it extremely difficult to reduce
protrudin levels in neurons --- either by shRNA or CRISPR --- without
killing the cells. We were able to study loss of protrudin expression
through overexpression of mutants with dominant negative effects, rather
than protrudin deletion or silencing.

\hypertarget{cellular-morphology}{%
\subsubsection{Cellular morphology}\label{cellular-morphology}}

While protrudin's role in cell outgrowth is interesting from the
perspective of axon regeneration, it could potentially play a number of
other roles in neuron morphology. Petrova et al.~showed that endogenous
protrudin is present in both axons and dendrites, but is most abundant
in the dendrites and cell bodies of mature neurons (Petrova et al.
2020). Consequently, protrudin's endogenous functions may well be more
apparent in the somatodendritic domain. In fact, there are a number of
published findings that suggest that protrudin may function in
dendrites, either to control dendritic spine morphology or support
synaptic function.

For example, the mechanism through which protrudin promotes invadopodia
formation in cancer cells is functionally similar to how lysosomal
exocytosis induces dendritic spine expansion, raising the possibility
that protrudin is involved in spine growth (Pedersen et al. 2020;
Padamsey et al. 2017). However, Pedersen et al.~found that this process
is dependent on FYCO1 as well as protrudin, which (as discussed in
section @ref(no-FYCO)), is largely absent in CNS neurons (Pedersen et
al. 2020). On the other hand, protrudin has already been shown to play a
functional role at dendritic spines, mediating long term depression by
allowing internalised AMPA receptors to be removed from the spine
through association with KIF5 (Brachet et al. 2021). In addition, ER
tubules (or the more complicated spine apparatus) have been observed
either transiently or permanently in most spines on hippocampal neurons
(Perez-Alvarez et al. 2020), which also suggests that protrudin is
likely to be present in or near spines and could act on their
morphology. Given protrudin's effect in cellular outgrowth, therefore,
it is reasonable to expect that protrudin overexpression might have an
effect on dendritic spine morphology.

\hypertarget{functional-effects}{%
\subsubsection{Functional effects}\label{functional-effects}}

To further examine protrudin's endogenous roles beyond cellular
outgrowth, we turned to the literature to identify published proteomics
data sets. We identified three studies that published lists of
protrudin-interacting proteins (Elbaz-Alon et al. 2020; Hashimoto et al.
2014; Matsuzaki et al. 2011). Each study used a different model system
and focused their analysis on different pathways, providing a variety of
perspectives.

Elbaz-Alon et al.~looked at protrudin in a human epithelial cell line,
where they investigated how protrudin interacts with PDZD8 and
contributes to ER contact sites (Elbaz-Alon et al. 2020). Hashimoto et
al.~used mouse brain neurons to look at interactions between protrudin
and HSP-related proteins (Hashimoto et al. 2014), while Matsuzaki et
al.~used a mouse neuroblast cell line to study the interaction between
protrudin and KIF5 (Matsuzaki et al. 2011). Individually, these reveal a
number of interesting interactions, indicating diverse roles for
protrudin that include but are not limited to axonal function. Taken
together, a substantial level of overlap between the interaction sets
reveals a conserved set of cellular functions that may help explain
protrudin's role in neurons.

\hypertarget{methods}{%
\subsection{Methods}\label{methods}}

General cell culture and microscopy methods can be found in section
@ref(METHODS).

\hypertarget{dendritic-spine-imaging-and-analysis}{%
\subsubsection{Dendritic spine imaging and
analysis}\label{dendritic-spine-imaging-and-analysis}}

Neurons were co-transfected with mCherry or an mCherry-tagged protrudin
construct and EGFP/Emerald. Cells were imaged as Z-stacks at DIV15,
either on the spinning disc at 100x, or confocal at 63x (see
@ref(METHODS) for microscope details). Image stacks from the green
channel were converted to maximum intensity projections, and a custom
ImageJ macro was used to select and crop 20\(\mu\)m sections from a
dendrite. These images were then blinded, and the Cell Counter Fiji
plugin was used to count and classify spines. For spine length, the Fiji
segmented line tool was used to measure along each spine.

\hypertarget{gene-ontology-analysis}{%
\subsubsection{Gene ontology analysis}\label{gene-ontology-analysis}}

Lists of protrudin-interacting proteins were obtained from supplementary
data published with (Elbaz-Alon et al. 2020), (Hashimoto et al. 2014),
and (Matsuzaki et al. 2011). Gene names were standardised across
experiments (and between human and mouse data sets) using Bioconductor
genome annotations (\texttt{org.Hs.eg.db} version 3.15.0 and
\texttt{org.Mm.eg.db} version 3.15.0). Gene ontology analysis for
biological process enrichment was carried out using the \texttt{limma}
Bioconductor package in R.

\hypertarget{hela-cell-culture}{%
\subsubsection{HeLa cell culture}\label{hela-cell-culture}}

HeLa cells were expanded before use and cryopreserved in individual
vials to prevent genetic drift between batches.

HeLa cells were grown in DMEM \emph{(Thermo Fisher, 41966-029)},
supplemented with 1\% FBS \emph{(Thermo Fisher, 10500064)} and
antibiotic-antimycotic \emph{(Thermo Fisher, 15240062)}, at 37°C, 5\%
CO\textsubscript{2}. For each batch of cells, a single vial was thawed,
passaged, and grown to 60-80\% confluence before transfection. Cells
were transfected with EGFP or EGFP-tagged protrudin under a CAG
promoter, using TransIT-LT1 reagent \emph{(Mirus Bio, MIR2300)},
according to the manufacturer's protocol. These were then passaged 24
hours after transfection to reduce overlap between cells and encourage
protrusion growth.

Cells were fixed with 3\% formaldehyde in PBS for 15 minutes at room
temperature. Immunochemistry was carried out according to the protocol
in section @ref(immuno). Staining with BODIPY TR C\textsubscript{5}
ceramide \emph{(Thermo Fisher, B34400)} was carried out according to the
manufacturer's protocol. Briefly, cells were incubated with 5\(\mu\)M
dye in DMEM for 30 minutes at 4°C. They were then recovered in
supplemented media at 37°C, 5\% CO\textsubscript{2} for 30 minutes
before fixing as usual.

\hypertarget{results}{%
\subsection{Results}\label{results}}

\hypertarget{protrudins-effect-on-dendritic-spine-number-length-and-morphology}{%
\subsubsection{Protrudin's effect on dendritic spine number, length, and
morphology}\label{protrudins-effect-on-dendritic-spine-number-length-and-morphology}}

Given protrudin's dendritic localisation and known effects on protrusion
formation, we first decided to examine whether protrudin plays a role in
regulating dendritic spine morphology. Dendritic spines show diverse
morphology within and between cells, which is generally considered to be
reflective of the maturity, functionality, and plasticity of each spine.
While these ultimately fall on a continuous spectrum of shapes, they are
conventionally grouped into four morphological classes --- mushroom,
thin, stubby, and filopodia (Pchitskaya and Bezprozvanny 2020). We
decided to use these classes to obtain an overview of spine morphology
on cells expressing protrudin.

We counted and manually classified dendritic spines on 20\(\mu\)m
sections of dendrite from DIV15 or DIV21 neurons co-expressing an
EGFP/Emerald filler with an mCherry-labelled protrudin construct (figure
@ref(fig:spine-type-full-figure)). These included WT and active
protrudin, as well as a set of domain-deletion mutants to act as
dominant negatives. \(\Delta\)FFAT, \(\Delta\)RBD, and \(\Delta\)FYVE
each have a single binding motif deleted, \(\Delta\)KIF5 has both the CC
and FFAT domains deleted, to completely block binding to kinesin, and
\(\Delta\)ER mutant (\(\Delta\)TM1-3) has all transmembrane domains
deleted (Petrova et al. 2020). The DIV15 data suggests that WT protrudin
might be associated with fewer, stubbier spines, although the
significant comparisons varied for each spine classification, and is
inconsistent with the DIV21 data.

%\newgeometry{a4paper, top=1.5in, left=1.5in, right=1.5in, bottom = 3cm} %unnecessary as long as 1.5in margins overall

\begin{figure}
\centering
\includegraphics{../figures/full/spine-type-full-figure-1.png}
\caption{A) Schematic showing selection of dendrite sections for
analysis and basic spine morphology groups. One dendrite section was
analysed per cell. Scale bar is 10\(\mu\)m. B-C) Total spine counts in
20\(\mu\)m section for DIV15 and DIV21 groups. D-E) Quantification of
each spine type, normalised to the total number of spines in the section
measured. DIV15: n = 30 cells per condition across three independent
batches; DIV21: n = 10 cells per condition in one batch.}
\end{figure}

%\restoregeometry
%\setstretch{1.5} %not restored by \restoregeometry

As manual spine classification is an inherently subjective measure, and
did not show any clear associations, we decided to look instead at spine
length, which can be measured more objectively. We re-analysed the
initial DIV15 data, measuring the length of each spine from the edge of
the dendrite. This did yield some significant comparisons, both in spine
number and length, so we followed this up with a full replication, which
we analysed in the same way (figure @ref(fig:spine-length-full-figure)).
While the replication also produced statistically significant results,
they were wholly inconsistent between the two experiments.

\begin{figure}
\centering
\includegraphics{../figures/full/spine-length-full-figure-1.png}
\caption{A-B) Number of dendritic spines counted per 20\(\mu\)m section
in cells co-expressing EGFP/Emerald with an mCherry-tagged protrudin
construct. Replicate 1: n = 30 cells per condition across three
independent batches, except \(\Delta\)RBD n = 29, \(\Delta\)KIF5 n = 20,
\(\Delta\)FYVE n = 19 across two batches. Replicate 2: n = 40 cells per
condition, except WT and \(\Delta\)ER n = 39, across three independent
batches. C-D) Length of each spine measured, in \(\mu\)m.}
\end{figure}

Taken together, these results show that protrudin overexpression does
not substantially impact dendritic spine number, length, or morphology.
These measurements were highly variable between cells and between
replicates, and did not show any consistent patterns that would suggest
that protrudin plays any role in controlling spine outgrowth. This
obviously cannot rule out changes in spine function, but in any case
such changes do not seem to be mediated by morphological changes at the
micron scale.

\hypertarget{GO-analysis}{%
\subsubsection{Protrudin's interactions with other proteins in the
cell}\label{GO-analysis}}

As our results did not indicate a clear role for protrudin in regulating
dendritic spine morphology, we turned to proteomic analysis to identify
other potential cellular functions, either in the somatodendritic domain
or in axons.

We identified three published lists of protrudin-interacting proteins
from (Elbaz-Alon et al. 2020), (Hashimoto et al. 2014), and (Matsuzaki
et al. 2011). As these studies used distinct methods and model systems,
we were most interested in the interactions that were conserved across
sets. After standardising the data across sources, we found 46
protrudin-interacting proteins that were identified in at least two
studies. Gene ontology analysis (biological process enrichment) of this
set showed a high representation of proteins involved in vesicular
transport, and in particular Golgi-related transport (`intra-Golgi
vesicle mediated transport', `Golgi vesicle transport', `ER to Golgi
vesicle-mediated transport', etc.) (figure
@ref(fig:proteomics-full-figure)). In contrast to endosomal transport,
protrudin's role in Golgi-related transport has not been extensively
studied, and we decided to follow this up with further experiments.

\begin{figure}
\centering
\includegraphics{../figures/full/proteomics-full-figure-1.png}
\caption{A) Venn diagram showing overlap between published lists of
protrudin-interacting proteins. B) Top 20 hits of GO biological process
enrichment, of all proteins appearing in at least two protrudin
interaction sets, sorted by p-value.}
\end{figure}

\hypertarget{hela-screen-for-changes-in-organelle-and-protein-distribution}{%
\subsubsection{HeLa screen for changes in organelle and protein
distribution}\label{hela-screen-for-changes-in-organelle-and-protein-distribution}}

When protrudin is overexpressed in HeLa cells, it drives the growth of
long cellular protrusions that superficially resemble developing
neurites (Shirane and Nakayama 2006). While these cells are genetically
and functionally very distinct from neurons, this nevertheless provides
a useful model system for examining protrudin's interactions with other
cellular components. We used this system to screen for proteins that
colocalise with protrudin in these cellular outgrowths, to quickly
identify organelles or complexes that might be transported into axons
when protrudin is overexpressed in neurons.

Based on the results of our gene ontology analysis, we selected a panel
of antibodies to screen for colocalisation with protrudin in HeLa cells.
These were primarily focused on components of the ER exit site to Golgi
secretory pathway, but also included proteins involved in lipid
metabolism, signalling, and amino acid transport.

\begin{longtable}[]{@{}ll@{}}
\caption{Markers for HeLa screen.}\tabularnewline
\toprule()
Marker & Description \\
\midrule()
\endfirsthead
\toprule()
Marker & Description \\
\midrule()
\endhead
BCAP31 & ER chaperone \\
SURF4 & ERES cargo loading \\
MIA3 (TANGO1) & ERES cargo loading \\
SEC31A & COPII coat component \\
ERGIC53 & ERGIC marker \\
GS28 & Golgi SNARE \\
GLG1 & Golgi glycoprotein \\
USO1 (p115) & ER to Golgi vesicle targeting \\
LAMP1 & lysosome marker \\
mTOR & signalling kinase \\
ACSL3 & fatty acid metabolism \\
SLC38A1 & glutamine uptake transporter \\
LC3 & autophagosome marker \\
BODIPY TR C\textsubscript{5} ceramide & ceramide lipid dye \\
\bottomrule()
\end{longtable}

\hypertarget{er--and-golgi-related-organelles}{%
\paragraph{ER- and Golgi-related
organelles}\label{er--and-golgi-related-organelles}}

Of the ERES-related proteins, different markers showed different levels
of colocalisation with protrudin. BCAP31 and SURF4 were consistently
enriched in cellular protrusions, more so than at other areas of the
cell periphery. MIA3 and SEC31A colocalised well with protrudin in the
perinuclear area, but were found at lower levels in protrusions, and
were less consistent between cells. At this stage, it was not possible
to conclusively say whether these variations represented differences in
how ERES proteins were transported or in how effectively the individual
antibodies stained these components, so we decided to proceed with
BCAP31 and SURF4, which produced more clear results. (Figure
@ref(fig:HeLa-ERES-grid-full-figure))

\begin{figure}
\centering
\includegraphics{../figures/full/HeLa-ERES-grid-full-figure-1.png}
\caption{Representative images of HeLa cells expressing EGFP-tagged
protrudin constructs and stained for ER/ERES-related markers A) BCAP31,
B) SURF4, C) MIA3(TANGO1), and D) SEC31A. All scale bars are
20\(\mu\)m.}
\end{figure}

Of the Golgi-related proteins, GS28, GLG1, and USO1 clearly labelled the
perinuclear Golgi compartment, but were either entirely absent or found
only at very low levels in protrusions. ERGIC53, on the other hand, was
highly enriched in protrusions, in most cases well above the levels seen
in the cell body. This was the strongest concentration with any of the
markers used in this screen, and point to the ERGIC compartment not only
colocalising with protrudin, but specifically playing a role in the
growth of cellular protrusions. This marker was added to the set (along
with BCAP31 and SURF4) to examine in neurons. (Figure
@ref(fig:HeLa-Golgi-grid-full-figure))

\begin{figure}
\centering
\includegraphics{../figures/full/HeLa-Golgi-grid-full-figure-1.png}
\caption{Representative images of HeLa cells expressing EGFP-tagged
protrudin constructs and stained for ERGIC and Golgi-related markers A)
ERGIC53, B) GS28, C) GLG1, and D) USO1(p115) E) Close up of ERGIC53
staining in the protrusion of a cell expressing WT protrudin, as shown
in panel A. F) Fluorescence profile along the protrusion shown in panel
E. All scale bars are 20\(\mu\)m.}
\end{figure}

\hypertarget{lysosomes-enzymes-and-transporters}{%
\paragraph{Lysosomes, enzymes, and
transporters}\label{lysosomes-enzymes-and-transporters}}

LAMP1 and mTOR, which we expected to move to the cell periphery in
response to protrudin overexpression, were present in protrusions at
relatively low levels. While protrudin overexpression did result in more
puncta away from the perinuclear cluster, this effect was not especially
dramatic, and the highest concentrations were still found near the cell
nucleus. This is in contrast with (Hong et al. 2017), which found a
stronger movement away from the nucleus when overexpressing Myc-tagged
protrudin. However, their culture protocol differed from ours in adding
2mM supplemental glutamine to the culture media, so the apparent
differences could reflect the amino acid dependence of the
protrudin-FYCO1 transport mechanism. (Figure
@ref(fig:HeLa-misc-grid-full-figure), panels A-B)

ACSL3 was found in cellular protrusions, but --- similar to MIA3 and
SEC31A --- was generally at low levels, and varied considerably between
cells. This may reflect on the antibody used, or may represent more
complicated cell-state dependent interaction with other metabolic
pathways. This was not taken forwards. (Figure
@ref(fig:HeLa-misc-grid-full-figure), panel C)

Finally, SLC38A1 was found to concentrate in protrusions, and similarly
to ERGIC53 was found at higher levels here than near the nucleus. This
result is particularly interesting, as SLC38A1 is a potential cargo of
the local secretory pathway. Matsuzaki et al.~identified SLC38A1 in
their set of protrudin-interacting proteins, and Shigeoka et al.~showed
that it was locally translated in RGC axons (Matsuzaki et al. 2011;
Shigeoka et al. 2016). SLC38A1 was added to the list with BCAP31, SURF4,
and ERGIC53 to examine in neurons. (Figure
@ref(fig:HeLa-misc-grid-full-figure), panel D)

\begin{figure}
\centering
\includegraphics{../figures/full/HeLa-misc-grid-full-figure-1.png}
\caption{Representative images of HeLa cells expressing EGFP-tagged
protrudin constructs and stained for A) LAMP1, B) mTOR, C) ACSL3, and D)
SLC38A1. All scale bars are 20\(\mu\)m.}
\end{figure}

\hypertarget{other-morphological-changes}{%
\paragraph{Other morphological
changes}\label{other-morphological-changes}}

In addition to protrusions, we noted that protrudin overexpression also
induces the formation of large, ring-shaped structures in the perinucear
region of HeLa cells. These are observed occasionally in neurons
expressing protrudin, but not to the same extent as in HeLa cells. While
these were not the primary focus of this experiment, they did reveal
interesting results with two of the markers screened, and may indicate
protrudin involvement in additional cellular pathways.

LC3 strongly colocalised with protrudin in these structures, suggesting
that they may represent phagophores or autophagosomes (figure
@ref(fig:LC3-full-figure)). Notably, LC3 is \emph{not} enriched in
cellular protrusions (panel C), in contrast to the markers discussed
above. This may represent a different pathway activated by protrudin
overexpression. These LC3-labelled structures are also smaller and less
abundant in cells expressing only EGFP, suggesting that their formation
is influenced by protrudin overexpression.

We followed up this result with a pilot experiment in neurons,
co-expressing EGFP-protrudin and mRFP1-tagged LC3. LC3 showed
differences in localisation between cells, ranging from fully
cytoplasmic to fully punctate, which likely reflects the conversion
between soluble LC3-I and membrane-associated LC3-II (Kabeya et al.
2000). All cells expressing either WT or active protrudin contained at
least some LC3 puncta, in contrast to control cells, of which 12.5\% had
fully cytoplasmic LC3. This suggests that protrudin might be encouraging
autophagosome biogenesis, although further experiments are needed to
confirm this result.

\begin{figure}
\centering
\includegraphics{../figures/full/LC3-full-figure-1.png}
\caption{A) Representative image of HeLa cells expressing WT protrudin
and stained for LC3. Inset is 20x20\(\mu\)m. B) Fluorescence profile
along the dotted line shown in panel A inset. Note strong correspondence
between protrudin (green) and LC3 (magenta) signal. C) Protrusion of a
HeLa cell expressing WT protrudin and stained for LC3. Note the lack of
LC3 enrichment with protrudin in the tip, compared to the cell body.
Scale bar is 20\(\mu\)m. D) Representative images of neurons expressing
mRFP1-tagged LC3 showing range of distribution patterns. Scale bars are
20\(\mu\)m. E) Quantification of LC3 distributions in cells
co-expressing EGFP-protrudin and mRFP1-LC3. n = 40 cells per condition
across two independent batches.}
\end{figure}

BODIPY C\textsubscript{5} ceramide, a lipid dye often used to label the
Golgi apparatus, strongly labelled the lumen of these organelles. It is
not clear in this context whether the dye is labelling Golgi-related
structures or lipid droplets, however it is worth noting that other
Golgi markers did \emph{not} colocalise with these structures
(e.g.~figure @ref(fig:HeLa-Golgi-grid-full-figure), particularly visible
in panel C). Given the relationship with LC3 as discussed above, this
may indicate an increase in lipophagy, although further experiments
would be needed to confirm this hypothesis.

\begin{figure}
\centering
\includegraphics{../figures/full/ceramide-full-figure-1.png}
\caption{A) Representative image of HeLa cells expressing WT protrudin
and stained with BODIPY C\textsubscript{5} ceramide dye. Inset is
20x20\(\mu\)m. B) Fluorescence profile along the dotted line shown in
panel A inset. Note BODIPY signal (magenta) between spikes of protrudin
(green).}
\end{figure}

\hypertarget{discussion}{%
\subsection{Discussion}\label{discussion}}

\hypertarget{protrudin-and-dendritic-spines}{%
\subsubsection{Protrudin and dendritic
spines}\label{protrudin-and-dendritic-spines}}

Protrudin overexpression --- or inhibition through the expression of
dominant negative mutants --- did not substantially influence dendritic
spine morphology. Considering the mechanism described in Pedersen et
al., it seems likely that this is another result of the absence of FYCO1
in CNS neurons (Pedersen et al. 2020). We cannot, however, rule out an
effect on dendritic spine function, and more work would be required to
understand whether protrudin overexpression may influence neuronal
signalling.

\hypertarget{the-eres-to-golgi-pathway}{%
\subsubsection{The ERES to Golgi
pathway}\label{the-eres-to-golgi-pathway}}

In non-polarised cells, membrane proteins are typically translated in
the ER, released from ERES, trafficked to the Golgi via the ERGIC
compartment, and secreted in post-Golgi carriers. In neurons, several
alternate pathways exist, to account for the local translation needs of
compartments distant from the cell body. In dendrites, for example,
Golgi outposts and satellites serve some of the same functions as the
main Golgi apparatus, acting as local points for protein glycosylation
(Mikhaylova et al. 2016). The ERGIC compartment is also involved in a
Golgi-bypassing secretory pathway, which has been observed in dendrites
(Bowen et al. 2017).

There is evidence for the local translation of integral membrane
proteins in CNS axons (Shigeoka et al. 2016), although it is not
currently understood how they are trafficked from the ER to the cell
surface. Recent work in DRG neurons has identified a non-conventional
secretory pathway involving Golgi satellites and LAMP1-labelled
organelles, which plays a role in the delivery of TRPM8 ion channels
(Cornejo et al. 2020). It is possible that a similar pathway functions
in CNS axons, providing a mechanism through which protrudin expression
could promote secretion and growth in the absence of any changes to the
cell body Golgi apparatus.

It is worth noting that three of the four proteins identified in our
HeLa screen --- ERGIC53, SURF4, and BCAP31 --- have been identified in
the ERGIC compartment, although there is some indication that BCAP31
localisation varies by cell type (Breuza et al. 2004; Quistgaard 2021).
This result suggests a novel role for this compartment in
protrudin-driven protrusion formation, and potentially in axon
regeneration, which is explored further in the next chapter.

\hypertarget{refs}{}
\begin{CSLReferences}{1}{0}
\leavevmode\vadjust pre{\hypertarget{ref-bowenGolgiindependentSecretoryTrafficking2017}{}}%
Bowen, Aaron B, Ashley M Bourke, Brian G Hiester, Cyril Hanus, and
Matthew J Kennedy. 2017. {``Golgi-Independent Secretory Trafficking
Through Recycling Endosomes in Neuronal Dendrites and Spines.''} Edited
by Graeme W Davis. \emph{eLife} 6 (September): e27362.
\url{https://doi.org/10.7554/eLife.27362}.

\leavevmode\vadjust pre{\hypertarget{ref-brachetKinesin1protrudinComplex2021}{}}%
Brachet, Anna, Argentina Lario, Alba Fernández-Rodrigo, Frank F.
Heisler, Yolanda Gutiérrez, Clara Lobo, Matthias Kneussel, and José A.
Esteban. 2021. {``A Kinesin 1-Protrudin Complex Mediates {AMPA} Receptor
Synaptic Removal During Long-Term Depression.''} \emph{Cell Reports} 36
(5): 109499. \url{https://doi.org/10.1016/j.celrep.2021.109499}.

\leavevmode\vadjust pre{\hypertarget{ref-breuzaProteomicsEndoplasmicReticulumGolgi2004}{}}%
Breuza, Lionel, Regula Halbeisen, Paul Jenö, Stefan Otte, Charles
Barlowe, Wanjin Hong, and Hans-Peter Hauri. 2004. {``Proteomics of
{Endoplasmic Reticulum-Golgi Intermediate Compartment} ({ERGIC})
{Membranes} from {Brefeldin A-treated HepG2 Cells Identifies ERGIC-32},
a {New Cycling Protein That Interacts} with {Human Erv46} *.''}
\emph{Journal of Biological Chemistry} 279 (45): 47242--53.
\url{https://doi.org/10.1074/jbc.M406644200}.

\leavevmode\vadjust pre{\hypertarget{ref-cornejoNonconventionalAxonalOrganelles2020}{}}%
Cornejo, Víctor Hugo, Carolina González, Matías Campos, Leslie
Vargas-Saturno, María de los Ángeles Juricic, Stéphanie Miserey-Lenkei,
María Pertusa, Rodolfo Madrid, and Andrés Couve. 2020.
{``Non-Conventional {Axonal Organelles Control Trpm8 Ion Channel
Trafficking} and {Peripheral Cold Sensing}.''} \emph{Cell Reports} 30
(13): 4505--4517.e5. \url{https://doi.org/10.1016/j.celrep.2020.03.017}.

\leavevmode\vadjust pre{\hypertarget{ref-elbaz-alonPDZD8InteractsProtrudin2020}{}}%
Elbaz-Alon, Yael, Yuting Guo, Nadav Segev, Michal Harel, Daniel E.
Quinnell, Tamar Geiger, Ori Avinoam, Dong Li, and Jodi Nunnari. 2020.
{``{Pdzd8} Interacts with {Protrudin} and {Rab7} at {ER-late} Endosome
Membrane Contact Sites Associated with Mitochondria.''} \emph{Nature
Communications} 11 (1): 3645.
\url{https://doi.org/10.1038/s41467-020-17451-7}.

\leavevmode\vadjust pre{\hypertarget{ref-hashimotoProtrudinRegulatesEndoplasmic2014}{}}%
Hashimoto, Yutaka, Michiko Shirane, Fumiko Matsuzaki, Shotaro Saita,
Takafumi Ohnishi, and Keiichi I. Nakayama. 2014. {``Protrudin Regulates
Endoplasmic Reticulum Morphology and Function Associated with the
Pathogenesis of Hereditary Spastic Paraplegia.''} \emph{The Journal of
Biological Chemistry} 289 (19): 12946--61.
\url{https://doi.org/10.1074/jbc.M113.528687}.

\leavevmode\vadjust pre{\hypertarget{ref-hongPtdIns3PControlsMTORC12017}{}}%
Hong, Zhi, Nina Marie Pedersen, Ling Wang, Maria Lyngaas Torgersen,
Harald Stenmark, and Camilla Raiborg. 2017. {``{PtdIns3P} Controls
{mTORC1} Signaling Through Lysosomal Positioning.''} \emph{Journal of
Cell Biology} 216 (12): 4217--33.
\url{https://doi.org/10.1083/jcb.201611073}.

\leavevmode\vadjust pre{\hypertarget{ref-kabeyaLC3MammalianHomologue2000}{}}%
Kabeya, Yukiko, Noboru Mizushima, Takashi Ueno, Akitsugu Yamamoto,
Takayoshi Kirisako, Takeshi Noda, Eiki Kominami, Yoshinori Ohsumi, and
Tamotsu Yoshimori. 2000. {``{Lc3}, a Mammalian Homologue of Yeast
{Apg8p}, Is Localized in Autophagosome Membranes After Processing.''}
\emph{The EMBO Journal} 19 (21): 5720--28.
\url{https://doi.org/10.1093/emboj/19.21.5720}.

\leavevmode\vadjust pre{\hypertarget{ref-matsuzakiProtrudinServesAdaptor2011}{}}%
Matsuzaki, Fumiko, Michiko Shirane, Masaki Matsumoto, and Keiichi I.
Nakayama. 2011. {``Protrudin Serves as an Adaptor Molecule That Connects
{Kif5} and Its Cargoes in Vesicular Transport During Process
Formation.''} \emph{Molecular Biology of the Cell} 22 (23): 4602--20.
\url{https://doi.org/10.1091/mbc.e11-01-0068}.

\leavevmode\vadjust pre{\hypertarget{ref-mikhaylovaGS2016}{}}%
Mikhaylova, Marina, Sujoy Bera, Oliver Kobler, Renato Frischknecht, and
Michael R. Kreutz. 2016. {``A {Dendritic Golgi Satellite} Between
{ERGIC} and {Retromer}.''} \emph{Cell Reports} 14 (2): 189--99.
\url{https://doi.org/10.1016/j.celrep.2015.12.024}.

\leavevmode\vadjust pre{\hypertarget{ref-padamseyActivityDependentExocytosisLysosomes2017}{}}%
Padamsey, Zahid, Lindsay McGuinness, Scott J. Bardo, Marcia Reinhart,
Rudi Tong, Anne Hedegaard, Michael L. Hart, and Nigel J. Emptage. 2017.
{``Activity-{Dependent Exocytosis} of {Lysosomes Regulates} the
{Structural Plasticity} of {Dendritic Spines}.''} \emph{Neuron} 93 (1):
132--46. \url{https://doi.org/10.1016/j.neuron.2016.11.013}.

\leavevmode\vadjust pre{\hypertarget{ref-pchitskayaDendriticSpinesShape2020}{}}%
Pchitskaya, Ekaterina, and Ilya Bezprozvanny. 2020. {``Dendritic {Spines
Shape Analysis}\textemdash{{Classification}} or {Clusterization}?
{Perspective}.''} \emph{Frontiers in Synaptic Neuroscience} 0.
\url{https://doi.org/10.3389/fnsyn.2020.00031}.

\leavevmode\vadjust pre{\hypertarget{ref-pedersenProtrudinmediatedEREndosome2020}{}}%
Pedersen, Nina Marie, Eva Maria Wenzel, Ling Wang, Sandra Antoine,
Philippe Chavrier, Harald Stenmark, and Camilla Raiborg. 2020.
{``Protrudin-Mediated {ER}\textendash endosome Contact Sites Promote
{Mt1-MMP} Exocytosis and Cell Invasion.''} \emph{Journal of Cell
Biology} 219 (8): e202003063.
\url{https://doi.org/10.1083/jcb.202003063}.

\leavevmode\vadjust pre{\hypertarget{ref-perez-alvarezEndoplasmicReticulumVisits2020}{}}%
Perez-Alvarez, Alberto, Shuting Yin, Christian Schulze, John A. Hammer,
Wolfgang Wagner, and Thomas G. Oertner. 2020. {``Endoplasmic Reticulum
Visits Highly Active Spines and Prevents Runaway Potentiation of
Synapses.''} \emph{Nature Communications} 11 (1): 5083.
\url{https://doi.org/10.1038/s41467-020-18889-5}.

\leavevmode\vadjust pre{\hypertarget{ref-petrovaProtrudinFunctionsEndoplasmic2020}{}}%
Petrova, Veselina, Craig S. Pearson, Jared Ching, James R. Tribble,
Andrea G. Solano, Yunfei Yang, Fiona M. Love, et al. 2020. {``Protrudin
Functions from the Endoplasmic Reticulum to Support Axon Regeneration in
the Adult {CNS}.''} \emph{Nature Communications} 11 (1): 5614.
\url{https://doi.org/10.1038/s41467-020-19436-y}.

\leavevmode\vadjust pre{\hypertarget{ref-quistgaardBAP31PhysiologicalFunctions2021}{}}%
Quistgaard, Esben M. 2021. {``{Bap31}: {Physiological} Functions and
Roles in Disease.''} \emph{Biochimie} 186 (July): 105--29.
\url{https://doi.org/10.1016/j.biochi.2021.04.008}.

\leavevmode\vadjust pre{\hypertarget{ref-shigeokaDynamicAxonalTranslation2016}{}}%
Shigeoka, Toshiaki, Hosung Jung, Jane Jung, Benita Turner-Bridger,
Jiyeon Ohk, Julie Qiaojin Lin, Paul S. Amieux, and Christine E. Holt.
2016. {``Dynamic {Axonal Translation} in {Developing} and {Mature Visual
Circuits}.''} \emph{Cell} 166 (1): 181--92.
\url{https://doi.org/10.1016/j.cell.2016.05.029}.

\leavevmode\vadjust pre{\hypertarget{ref-shiraneProtrudinInducesNeurite2006}{}}%
Shirane, Michiko, and Keiichi I. Nakayama. 2006. {``Protrudin {Induces
Neurite Formation} by {Directional Membrane Trafficking}.''}
\emph{Science} 314 (5800): 818--21.
\url{https://doi.org/10.1126/science.1134027}.

\end{CSLReferences}

\end{document}
